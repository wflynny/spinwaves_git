\documentclass{article}[11pt]


\author{Bill Flynn}
\title{Spinwave Calc File}


\begin{document}
\maketitle

%%%%%%%%%%%%%%%%%%%%%%%%%%%%%%%%%%%%%%%%%%%%%%%%%%%%%%%%%%%%%%%%%%%%%%%%%%%%%%%%%%%%%%%%%%%%%%%%%%%%
%%%%%%%%%%%%%%%%%%%%%%%%%%%%%%%%%%%%%%%%%%%%%%%%%%%%%%%%%%%%%%%%%%%%%%%%%%%%%%%%%%%%%%%%%%%%%%%%%%%%

\section{}

First we generate the spin of each atom in the local coordinate system (using the z-axis as the quantization axis). This yields a list of $\mbox{length}\;=\;\mbox{number of atoms}$ with elements of the form:

\begin{equation}
	s_{i}=\left(\begin{array}{ccc}
		\sqrt{\frac{S}{2}}\left(c_i + c_i^\dagger\right) \\
		\sqrt{\frac{S}{2}}\left(c_i - c_i^\dagger\right)/i \\
		\textbf{S}-c_i c_i^\dagger 
	\end{array}\right) 
\end{equation}

where $\textbf{S}$ is a constant (bolded for emphasis -- do not confuse with matrix/vector/etc.). We then transform the local spins into the global coordinate system using a specific rotation matrix determined during the simulated annealing process.

\begin{equation}
	S_{i} = Rot_{i} s_{i}
\end{equation}

Each of the N atoms in the atom array has a vector which describes its anisotropy, or preference to point in a particular direction, and a matrix which describes how that atom interacts with other atoms. We use these to create the hamiltonian:
\begin{equation}
	H = -\displaystyle \sum_{i,j=0}^{N} S_{i} J_{ij} S_{j} - \displaystyle \sum_{\alpha}D_{\alpha}S_{i_{\alpha}}^{2}
\end{equation}
\noindent where $J_{ij}$ is the interaction matrix between two interacting atoms and $D_{\alpha}$ corresponds to preference along the $\left\{x,y,z\right\}$ directions. This process yeilds an expression which we then simplify by removing all terms that do not depend on spin, $\textbf{S}$. For example, the expression
\begin{equation}
	A\textbf{S}^{2} + B \textbf{S} + C + D \textbf{S}^{2} + E + F \textbf{S} + \ldots
\end{equation}
is reduced to 
\begin{equation}
	A\textbf{S}^{2} + B \textbf{S} + D \textbf{S}^{2} + F \textbf{S} + \ldots.
\end{equation}

For the Fourier transform, we make the following substitutions in the Hamiltonian, $H$:
\begin{eqnarray}
	c_{i}^{\dagger}c_{j}^{\dagger} & \rightarrow & \frac{1}{2}\left(c_{k,i}^{\dagger}c_{-k,j}^{\dagger}e^{-i \vec{k} \cdot \vec{r}} + c_{-k,i}^{\dagger}c_{k,j}^{\dagger}e^{i \vec{k} \cdot \vec{r}}\right) \\
	c_{i}c_{j} & \rightarrow & \frac{1}{2}\left(c_{k,i}c_{-k,j}e^{i \vec{k} \cdot \vec{r}} + c_{-k,i}c_{k,j}e^{-i \vec{k} \cdot \vec{r}}\right) \\
	c_{i}^{\dagger}c_{j} & \rightarrow & \frac{1}{2}\left(c_{k,i}^{\dagger}c_{k,j}e^{-i \vec{k} \cdot \vec{r}} + c_{-k,i}^{\dagger}c_{-kj}e^{i \vec{k} \cdot \vec{r}}\right) \\
	c_{i}c_{j}^{\dagger} & \rightarrow & \frac{1}{2}\left(c_{k,i}c_{k,j}^{\dagger}e^{i \vec{k} \cdot \vec{r}} + c_{-k,i}c_{-k,j}^{\dagger}e^{-i \vec{k} \cdot \vec{r}}\right) \\	
	c_{j}^{\dagger}c_{j} & \rightarrow & \frac{1}{2}\left(c_{k,j}^{\dagger}c_{k,j}+c_{-k,j}^{\dagger}c_{-k,j}\right)
\end{eqnarray}
\noindent Additionally, we substitute in the following commutation relations:
\begin{eqnarray}
	c_{k,i}c_{k,j}^{\dagger} & \rightarrow & c_{k,j}^{\dagger}c_{k,i} + \delta_{ij} \\
	c_{-k,i}^{\dagger}c_{-k,j} & \rightarrow & c_{-k,j}^{\dagger}c_{-k,i} + \delta_{ij} \\	
	c_{k,i}c_{-k,j} & \rightarrow & c_{-k,j}c_{k,i} \\
	c_{-k,j}^{\dagger}c_{k,i}^{\dagger} & \rightarrow & c_{k,i}^{\dagger}c_{-k,j}^{\dagger}
\end{eqnarray}

We then create the two arrays
\begin{equation}
	O = \left[ c_{k,1}, c_{k,2}, \ldots , c_{k,N}, c_{-k,1}^{\dagger}, c_{-k,2}^{\dagger}, \ldots, c_{-k,N}^{\dagger} \right]
\end{equation}
and 
\begin{equation}
	O^{\dagger} = \left[ c_{k,1}^{\dagger}, c_{k,2}^{\dagger}, \ldots , c_{k,N}^{\dagger}, c_{-k,1}, c_{-k,2}, \ldots, c_{-k,N} \right]
\end{equation}
which we use to rewrite the Hamiltonian: 
\begin{equation}
	H = X_{ij} O_{i} O_{j}^{\dagger}
\end{equation}
We take these coefficients, the $X_{ij}$, and construct the matrix $X$ from them. We also construct a matrix $G$ that is the same size as $X$ which has a diagonal that looks like $\left\{1,1,\ldots,1,-1,-1,\ldots,-1\right\}$. The product $G X$ is our diagonalized Hamiltonian.  

\end{document}